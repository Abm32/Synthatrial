\documentclass[journal,transmag]{IEEEtran}

\hyphenation{op-tical net-works semi-conduc-tor}
\usepackage{fancyhdr}
\usepackage{url}

% --- Header and Footer Configuration ---
\fancypagestyle{plain}{%
    \fancyhf{}
    \fancyhead[R]{Samanwaya'26}
    \fancyfoot[C]{Abhimanyu R. B. \quad | \quad \thepage \quad | \quad Anukriti: In Silico Pharmacogenomics}
    \renewcommand{\headrulewidth}{0pt}
    \renewcommand{\footrulewidth}{0pt}
}
\pagestyle{fancy}
\fancyhf{}
\fancyhead[R]{SAMANWAYA'26}
\fancyfoot[L]{Abhimanyu R. B.}
\fancyfoot[C]{\thepage}
\fancyfoot[R]{Anukriti: In Silico Pharmacogenomics}

\renewcommand{\headrulewidth}{0pt}
\renewcommand{\footrulewidth}{0pt}

\begin{document}

% --- Title ---
\title{Anukriti: An Agentic AI Framework for In Silico Pharmacogenomics and Pre-Clinical Safety Screening}

% --- Authors ---
\author{\IEEEauthorblockN{Abhimanyu R. B.\IEEEauthorrefmark{1}}
\IEEEauthorblockA{\IEEEauthorrefmark{1}Department of Computer Science and Engineering, \\
College of Engineering Karunagappally, Kerala, India \\
Email: abhimanyurbsa@gmail.com}
}

\markright{Samanwaya'26}

\IEEEtitleabstractindextext{%

% --- Abstract ---
\begin{abstract}
Clinical drug discovery currently faces a 90\% failure rate, largely attributed to the lack of genetic diversity in early-stage safety testing. Traditional clinical trials often rely on homogenous population data, leading to unforeseen Adverse Drug Reactions (ADRs) when drugs are released to diverse global populations. This paper presents \textbf{Anukriti}, a novel \textit{in silico} pharmacogenomics platform that utilizes Agentic Artificial Intelligence to simulate drug interactions against diverse ``Synthetic Patient Cohorts.'' Unlike traditional computational biology methods that require massive supercomputing resources for molecular physics simulations, Anukriti leverages a lightweight, data-driven approach. The system integrates \textbf{Retrieval-Augmented Generation (RAG)} with \textbf{Large Language Models (LLMs)} and \textbf{Cheminformatics (RDKit)}. It converts chemical structures (SMILES) into vector embeddings, retrieves relevant biological interactions from the ChEMBL database, and employs AI agents to reason through metabolic pathways based on genomic data (VCF files). We demonstrate a working prototype validated on \textbf{Chromosome 22} of the 1000 Genomes Project dataset. The system achieved 100\% concordance with CPIC (Clinical Pharmacogenetics Implementation Consortium) guidelines in predicting risk levels for known CYP2D6 substrates (codeine, tramadol, metoprolol) against poor metabolizer profiles. An ablation study demonstrated that RAG retrieval significantly enhances mechanistic reasoning quality compared to LLM-only configurations. Additionally, the system demonstrated reasoning capability over unseen compounds (e.g., dextromethorphan) not explicitly covered by CPIC guidelines, illustrating its potential for generalizing pharmacogenomic decision-support to novel compounds. This study illustrates the potential of Agentic AI to democratize pre-clinical safety screening, allowing for scalable, population-specific toxicity prediction before physical trials begin.
\end{abstract}

% --- Keywords ---
\begin{IEEEkeywords}
Agentic AI, Pharmacogenomics, In Silico Trials, Large Language Models (LLM), Drug Safety, RAG.
\end{IEEEkeywords}}

\maketitle

\IEEEdisplaynontitleabstractindextext
\IEEEpeerreviewmaketitle

% --- Introduction ---
\section{Introduction}
\IEEEPARstart{T}{he} pharmaceutical industry faces a critical efficiency crisis. Developing a new therapeutic typically costs over \$2 billion and spans more than a decade, yet approximately 90\% of drug candidates fail during clinical trials \cite{hinton2018}. A significant portion of these failures stems from safety issues and adverse drug reactions (ADRs) that were not identified during pre-clinical testing.

A primary cause of this oversight is the ``genetic blind spot'' in traditional drug discovery. Historically, genomic data used for drug validation has been heavily biased toward populations of European ancestry \cite{genomemed}. When these drugs are administered to a global population with diverse genetic markers---such as specific HLA alleles common in Asian or African demographics---unexpected toxicity often occurs.

Current methods to address this include \textit{in vivo} animal testing, which raises ethical concerns and often fails to predict human responses, and physics-based molecular simulations, which are computationally prohibitive for large-scale population screening.

In this paper, we propose \textbf{Anukriti} (Sanskrit for ``Simulation'' or ``Replica''), a scalable \textit{in silico} framework that utilizes Agentic Artificial Intelligence to simulate clinical trials on ``Synthetic Patient Cohorts.'' By combining Retrieval-Augmented Generation (RAG) with genomic data from the 1000 Genomes Project, Anukriti acts as a pre-clinical safety filter, predicting population-specific risks before physical testing begins.

This paper makes the following contributions: (1) a novel Agentic AI framework that integrates RAG with LLMs for pharmacogenomic risk prediction, enabling reasoning over unseen drug compounds through molecular similarity; (2) validation on real genomic data (1000 Genomes Project) demonstrating concordance with established clinical guidelines (CPIC), matching rule-based baseline performance while providing mechanistic reasoning; (3) a lightweight, scalable architecture that avoids the computational overhead of traditional molecular dynamics simulations; and (4) explicit prompt engineering with CPIC-based risk definitions that enable accurate distinction between High-risk (requiring alternative drugs) and Medium-risk (manageable with dose adjustment) scenarios.

% --- Related Work ---
\section{Related Work}
The concept of \textit{in silico} clinical trials has gained traction as a method to reduce regulatory barriers and costs \cite{viceconti2016}. Traditional approaches primarily rely on Quantitative Systems Pharmacology (QSP) and molecular dynamics simulations. While accurate, these methods require extensive computational power and are often limited to single-protein interactions.

Recent advancements in deep learning have introduced data-driven approaches. AlphaFold and ESMFold have revolutionized protein structure prediction \cite{jumper2021}, enabling static analysis of drug-target binding. However, these models do not inherently account for system-wide metabolic pathways or patient-specific genetic variations (pharmacogenomics).

The emergence of Large Language Models (LLMs) with reasoning capabilities offers a new avenue. Recent studies on ``Agentic AI'' \cite{xi2023} suggest that LLMs can utilize tools to perform complex multi-step reasoning. Anukriti builds upon this by integrating LLM agents with Retrieval-Augmented Generation (RAG) over structured biomedical knowledge bases to simulate dynamic physiological responses.

Existing pharmacogenomic decision-support systems primarily rely on rule-based mappings between known genetic variants and drug recommendations, lacking the ability to reason over novel compounds or complex multi-gene interactions. These systems require manual curation for each drug-gene pair and cannot generalize to compounds not present in their knowledge base. Anukriti addresses this limitation by leveraging molecular similarity and agentic reasoning to predict risks for drugs not explicitly covered by clinical guidelines.

% --- Methodology ---
\section{Proposed Methodology}
The Anukriti architecture is designed to be lightweight and modular, avoiding the need for high-performance computing clusters. The workflow consists of three primary stages: Input Processing, Vector Retrieval, and Agentic Simulation.

\subsection{System Architecture}
The system accepts two primary inputs: the chemical structure of the drug candidate (in SMILES format) and the genomic profile of a synthetic patient (derived from VCF files). The drug SMILES string is processed using RDKit to generate Morgan Fingerprints (radius 2, 2048 bits), which are then used for similarity search in a vector database (Pinecone) containing drug-target interaction data from ChEMBL. The retrieved similar drugs, along with patient genetic variants (e.g., CYP2D6 metabolizer status), are provided to an LLM-based agent (Gemini 2.5 Flash) that reasons through metabolic pathways using CPIC-guided risk definitions.

\subsection{Agentic Reasoning Engine}
The core of the framework is an LLM-based agent that receives: (1) the chemical properties of the new drug, (2) the retrieval context (side effects of chemically similar drugs from ChEMBL), and (3) the specific genetic variants of the synthetic patient (e.g., status of CYP enzymes). The LLM prompt is engineered with explicit risk level definitions based on CPIC guidelines, including examples of High-risk (complete lack of efficacy or significant toxicity requiring alternative drugs) and Medium-risk (manageable with dose adjustment) scenarios. Using Chain-of-Thought (CoT) prompting, the agent follows a structured reasoning process: (1) identifying whether the drug requires CYP2D6 for activation (prodrug) or clearance (direct substrate), (2) assessing the impact of poor metabolizer status, (3) determining severity (complete failure vs. manageable reduction), and (4) classifying risk level based on CPIC guidelines.

% --- Results ---
\section{Results and Discussion}
To validate the framework, we conducted a pilot study focusing on \textbf{Chromosome 22} of the 1000 Genomes Project dataset, which contains the CYP2D6 gene responsible for metabolizing approximately 25\% of all clinically used drugs.

\subsection{Validation Results}
We tested the system with known CYP2D6 substrates including codeine, tramadol, and metoprolol against poor metabolizer profiles. Both the rule-based CPIC baseline and Anukriti achieved perfect concordance (3/3 exact risk level matches) with CPIC recommendations. Anukriti correctly identified: (1) High-risk scenarios for codeine (reduced efficacy due to lack of activation to morphine) and metoprolol (increased toxicity from accumulation), and (2) Medium-risk scenario for tramadol (reduced activation, manageable with dose adjustment). All predictions aligned with CPIC guidelines, demonstrating the system's ability to accurately reason through pharmacogenomic interactions and distinguish between severe consequences requiring alternative drugs (High risk) and manageable reductions that can be addressed with dose adjustment (Medium risk).

\subsection{Ablation Study}
An ablation study comparing Anukriti with and without RAG retrieval demonstrated that while both configurations achieved CPIC concordance on risk level classification, RAG-enabled Anukriti produced significantly higher quality mechanistic reasoning. Without RAG, the agent exhibited reduced mechanistic specificity and increased ambiguity, relying primarily on training data knowledge. With RAG, the agent consistently referenced chemically similar compounds and known metabolic pathways, resulting in more stable and interpretable risk predictions grounded in established biological knowledge.

\subsection{Unseen Compound Reasoning}
To evaluate generalization beyond CPIC-covered drugs, we tested the system on dextromethorphan, a CYP2D6 substrate not explicitly referenced in CPIC guidelines. Using molecular similarity retrieval, the system identified known CYP2D6 substrates (codeine, tramadol) and reasoned through the metabolic pathway, inferring a Medium Risk classification based on reduced activation with alternative metabolic pathways available. This demonstrates Anukriti's ability to extend pharmacogenomic reasoning to compounds not explicitly covered by clinical guidelines, representing a key advantage over rule-based systems limited to pre-curated drug-gene pairs.

\subsection{Performance}
Performance evaluation demonstrated retrieval latency of 150-200ms for vector similarity search and full agentic simulation completing in 7-12 seconds (median 9.0s) per patient profile, including LLM API call overhead. This demonstrates scalability compared to traditional molecular dynamics simulations requiring days per simulation and high-performance computing clusters.

% --- Conclusion ---
\section{Conclusion}
Anukriti presents a novel approach to pre-clinical safety screening by combining cheminformatics with Agentic AI. The validation on Chromosome 22 demonstrates that LLMs, when grounded with correct genomic and chemical data and engineered with explicit risk level definitions, can achieve high concordance with established pharmacogenomic guidelines. The system's ability to distinguish between High-risk and Medium-risk scenarios, provide mechanistic reasoning, and generalize to unseen compounds demonstrates its clinical utility and represents a significant advancement over rule-based pharmacogenomic decision-support systems. Future work will focus on scaling to Whole Genome Sequencing data, implementing comprehensive star allele calling, and expanding validation to larger drug datasets.

% --- Acknowledgement ---
\section{Acknowledgement}
The author would like to acknowledge the open-source community for the development of RDKit and the 1000 Genomes Project consortium for making genomic data publicly accessible.

% --- References ---
\ifCLASSOPTIONcaptionsoff
  \newpage
\fi

\begin{thebibliography}{1}

\bibitem{hinton2018}
J.~Vamathevan, D.~Clark, P.~Czodrowski, et al., ``Applications of machine learning in drug discovery and development,'' \emph{Nature Reviews Drug Discovery}, vol.~18, no.~6, pp. 463--477, 2019.

\bibitem{genomemed}
S.~Sirugo, S.~M. Williams, and S.~A. Tishkoff, ``The missing diversity in human genetic studies,'' \emph{Cell}, vol.~177, no.~1, pp. 26--31, 2019.

\bibitem{viceconti2016}
M.~Viceconti, F.~Pappalardo, B.~Rodriguez, et al., ``In silico clinical trials: how computer simulation will transform the biomedical industry,'' \emph{International Journal of Clinical Trials}, vol.~3, no.~2, pp. 37--46, 2016.

\bibitem{jumper2021}
J.~Jumper et al., ``Highly accurate protein structure prediction with AlphaFold,'' \emph{Nature}, vol.~596, pp. 583--589, 2021.

\bibitem{xi2023}
Z.~Xi et al., ``The rise and potential of large language model based agents: A survey,'' \emph{arXiv preprint arXiv:2309.07864}, 2023.

\bibitem{mendez2019}
D.~Mendez et al., ``ChEMBL: towards direct deposition of bioassay data,'' \emph{Nucleic Acids Research}, vol.~47, no.~D1, pp. D930--D940, 2019.

\bibitem{cpic}
K.~E. Caudle et al., ``Clinical Pharmacogenetics Implementation Consortium (CPIC) guidelines for codeine therapy in the context of cytochrome P450 2D6 (CYP2D6) genotype,'' \emph{Clinical Pharmacology \& Therapeutics}, vol.~91, no.~2, pp. 321--326, 2012.

\bibitem{pharmgkb2023}
M.~W. Whirl-Carrillo et al., ``An evidence-based framework for evaluating pharmacogenomics knowledge for personalized medicine,'' \emph{Clinical Pharmacology \& Therapeutics}, vol.~110, no.~3, pp. 563--572, 2021.

\bibitem{aihealthcare2024}
A.~Rajkomar, J.~Dean, and I.~Kohane, ``Machine learning in medicine,'' \emph{New England Journal of Medicine}, vol.~380, no.~14, pp. 1347--1358, 2019.

\end{thebibliography}

\end{document}
